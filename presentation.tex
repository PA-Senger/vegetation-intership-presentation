\documentclass[10pt]{beamer}
\usepackage{graphicx}
\usepackage{adjustbox}
\usepackage{hyperref}
\usepackage{amsmath}
\usepackage{hyperref}
\usepackage{graphicx}
\usepackage{float}
\usepackage{caption}
\usepackage{listings}
\usepackage{xcolor}
\usepackage{multimedia}


\colorlet{punct}{red!60!black}
\definecolor{background}{RGB}{240, 248, 255}
\definecolor{delim}{RGB}{20,105,176}
\colorlet{numb}{magenta!60!black}

\lstdefinelanguage{json}{
    basicstyle=\ttfamily\footnotesize\color{black},
    numbers=left,
    numberstyle=\scriptsize,
    stepnumber=1,
    numbersep=8pt,
    showstringspaces=false,
    breaklines=true,
    frame=lines,
    backgroundcolor=\color{background},
    literate=
     *{0}{{{\color{numb}0}}}{1}
      {1}{{{\color{numb}1}}}{1}
      {2}{{{\color{numb}2}}}{1}
      {3}{{{\color{numb}3}}}{1}
      {4}{{{\color{numb}4}}}{1}
      {5}{{{\color{numb}5}}}{1}
      {6}{{{\color{numb}6}}}{1}
      {7}{{{\color{numb}7}}}{1}
      {8}{{{\color{numb}8}}}{1}
      {9}{{{\color{numb}9}}}{1}
      {:}{{{\color{punct}{:}}}}{1}
      {,}{{{\color{punct}{,}}}}{1}
      {\{}{{{\color{delim}{\{}}}}{1}
      {\}}{{{\color{delim}{\}}}}}{1}
      {[}{{{\color{delim}{[}}}}{1}
      {]}{{{\color{delim}{]}}}}{1},
}

\lstset{frame=single, showstringspaces=false, columns=fixed, basicstyle={\ttfamily}, commentstyle={\it}, numbers=left, tabsize=4}

\definecolor{codebackground}{RGB}{240, 248, 255}
\definecolor{codecomment}{RGB}{106,153,85}
\definecolor{codekeyword}{RGB}{30,30,255}
\definecolor{codestring}{RGB}{163,21,21}
\definecolor{codenumber}{RGB}{100,100,100}

\lstdefinestyle{modernstyle}{
    backgroundcolor=\color{codebackground},
    commentstyle=\color{codecomment},
    keywordstyle=\color{codekeyword},
    numberstyle=\tiny\color{codenumber},
    stringstyle=\color{codestring},
    basicstyle=\ttfamily\footnotesize\color{black},
    breakatwhitespace=false,
    breaklines=true,
    captionpos=b,
    keepspaces=true,
    numbers=left,
    numbersep=5pt,
    showspaces=false,
    showstringspaces=false,
    showtabs=false,
    tabsize=4
}

\lstset{style=modernstyle}

\usetheme{Copenhagen}
\usecolortheme{default}
\setbeamertemplate{navigation symbols}{}

\title[ExaMA WP1 Vegetation]{
  \includegraphics[width=0.8\textwidth]{images/logo-ufr.png}
  exa-MA WP1 - Vegetation}
\author[PA Senger]{Pierre-Antoine SENGER}

\begin{document}


\begin{frame}[plain]
    \begin{center}
    \begin{tabular}{c c c}
    \includegraphics[width=100px]{images/logo-irma.png} &
    \includegraphics[width=100px]{images/logo-inria.png} &
    \includegraphics[width=100px]{images/logo-hidalgo2.png} \\
    \includegraphics[width=100px]{images/logo-ufr.png} &
    \includegraphics[width=100px]{images/logo-cemosis.png} &
    \includegraphics[width=100px]{images/logo-numpex.png} \\
    \end{tabular}
    \end{center}
\end{frame}

\begin{frame}{Introduction}
  \titlepage
\end{frame}

\begin{frame}{Introduction}

	This project is part of a series conducted within the \textbf{exa-MA} project,
	which is a segment of the \textbf{Numpex} project section.

	\vspace{1em}

	\begin{itemize}
		\item Exa-MA WP1 - Vegetation
		\item Exa-MA WP1 - Terrain
		\item Exa-MA WP1 - Urban Building LOD-1
		\item Exa-MA WP1 - Urban Building LOD-2 and Kinetic
		\item Exa-MA WP1 - Performance and Scalability
	\end{itemize}
\end{frame}

\begin{frame}{Introduction}
	\begin{figure}
		\centering
		\includegraphics[width=0.8\textwidth]{images/exama_overview.png}
		\caption{Exa-MA project overview \cite{numpex}}
		\label{fig:figure1}
	\end{figure}
\end{frame}

\begin{frame}{Context}
\begin{figure}[h]
    \centering
    \begin{minipage}{0.49\textwidth}
        \centering
        \includegraphics[width=\textwidth]{images/heat-street.png}
        \caption{Thermal image of a street depicting heat distribution \cite{img:street_thermography}}
        \label{fig:figure1}
    \end{minipage}\hfill
    \begin{minipage}{0.49\textwidth}
        \centering
        \includegraphics[width=\textwidth]{images/tree-shade.png}
        \caption{Tree providing shade to a building \cite{img:TreeShade}}
        \label{fig:figure2}
    \end{minipage}
    % \caption{Overall caption for both figures}
\end{figure}
\end{frame}

\begin{frame}{Context: Primiray focus}
	\begin{figure}
		\centering
		\includegraphics[width=\textwidth]{images/strasbourg-mesh-2.png}
		\caption{3D Model of Strasbourg, France}
		\label{fig:figure1}
	\end{figure}
\end{frame}

\begin{frame}{Context: Adaptibility}
	\begin{figure}
		\centering
		\includegraphics[width=\textwidth]{images/mesh-manhattan-2.png}
		\caption{3D Model of Manhattan, New York \cite{img:NY}}
		\label{fig:figure1}
	\end{figure}
\end{frame}

\begin{frame}{Objectives}
	\Large
	\begin{itemize}
		\item \textbf{\textcolor{red}{Extracting}} \textbf{tree data} from \textbf{OpenStreetMap}
		\item \textbf{\textcolor{red}{Generating}} \textbf{3D tree models}
		\item \textbf{\textcolor{red}{Integrating}} \textbf{tree models} in the \textbf{terrain mesh}
		\item \textbf{\textcolor{red}{Optimizing}} \textbf{computational efficiency}
	\end{itemize}
\end{frame}

\begin{frame}{Methodology Steps}
	\Large
	\begin{itemize}
		\item Data Acquisition
		\item Generating Tree Library
		\item Scaling Trees
		\item Placing Trees
		\item Merging Meshes
		\item Parallelization
	\end{itemize}
\end{frame}

\begin{frame}{Data Acquisition}
	\begin{figure}[h]
		\centering
		\begin{minipage}{0.49\textwidth}
			\centering
			\includegraphics[width=\textwidth]{images/logo-openstreetmap.png}
			\caption{OpenStreetMap Logo}
			\label{fig:figure1}
		\end{minipage}\hfill
		\begin{minipage}{0.49\textwidth}
			\centering
			\includegraphics[width=\textwidth]{images/logo-curl.png}
			\caption{Curl Logo}
			\label{fig:figure2}
		\end{minipage}
		% \caption{Overall caption for both figures}
	\end{figure}
\end{frame}

\begin{frame}[fragile]{Data Acquisition: Query Class}
	\begin{lstlisting}[language=C++]
void perform_query(std::string bbox, bool verbose) {
	std::string query =
		"[out:json]; (node(" + bbox + ")[\"natural\"=\"tree\"];); out;";
	cpr::Response r = cpr::Post(
		cpr::Url{"http://overpass-api.de/api/interpreter"}, cpr::Body{query},
		cpr::Header{{"Content-Type", "application/x-www-form-urlencoded"}},
		cpr::Timeout{10000} // Set a timeout of 10 seconds
	);
}
	\end{lstlisting}
\end{frame}

\begin{frame}[fragile]{Data acquisition: .json output}
	\begin{lstlisting}[language=json]
{
  "type": "node",
  "id": 10162018740,
  "lat": 48.5850910,
  "lon": 7.7502624,
  "tags": {
	"circumference": "1.47655",
	"diameter_crown": "5",
	"genus": "Platanus",
	"height": "6",
	"leaf_cycle": "deciduous",
	"leaf_type": "broadleaved",
	"natural": "tree",
	"ref": "16401",
	"source": "data.strasbourg.eu - patrimoine_arbore",
	"source:date": "2022-01-02",
	"species": "Platanus acerifolia x",
	"species:wikidata": "Q24853030"
  }
}
	\end{lstlisting}
\end{frame}

\begin{frame}{Generating Tree Library}
	\begin{figure}
		\centering
		\includegraphics[width=\textwidth]{images/Different-types-of-the-trees-shape.png}
		\caption{Different tree shapes}
		\label{fig:figure1}
	\end{figure}
\end{frame}

\begin{frame}{Tree Modeling: lod 0}
	\begin{figure}[h]
		\centering
		\begin{minipage}{0.49\textwidth}
			\centering
			\includegraphics[width=\textwidth]{images/tree-trunk.png}
			\caption{Tree Trunk model}
			\label{fig:figure1}
		\end{minipage}\hfill
		\begin{minipage}{0.49\textwidth}
			\centering
			\includegraphics[width=\textwidth]{images/cone.png}
			\caption{Cone shaped Tree model}
			\label{fig:figure2}
		\end{minipage}
		% \caption{Overall caption for both figures}
	\end{figure}
\end{frame}

\begin{frame}{Tree Modeling: lod 0}
	\begin{figure}[h]
		\centering
		\begin{minipage}{0.49\textwidth}
			\centering
			\includegraphics[width=\textwidth]{images/oval.png}
			\caption{Oval shaped Tree model}
			\label{fig:figure1}
		\end{minipage}\hfill
		\begin{minipage}{0.49\textwidth}
			\centering
			\includegraphics[width=\textwidth]{images/round.png}
			\caption{Round shaped Tree model}
			\label{fig:figure2}
		\end{minipage}
		% \caption{Overall caption for both figures}
	\end{figure}
\end{frame}

\begin{frame}{Tree Modeling: lod 1,2 and 3}
	\begin{figure}
		\centering
		\includegraphics[width=\textwidth]{images/ginkgo-sketchup.png}
		\caption{3D model of a Ginkgo tree on Sketchup}
		\label{fig:figure1}
	\end{figure}
\end{frame}

\begin{frame}{Tree preprocessing}
	Remove trunk/branches, normalize and center
	\begin{figure}[h]
		\centering
		\begin{minipage}{0.3\textwidth}
			\centering
			\includegraphics[width=\textwidth]{images/tree-conifer.png}
			\caption{A preprocessed conifer tree}
			\label{fig:figure1}
		\end{minipage}\hfill
		\begin{minipage}{0.3\textwidth}
			\centering
			\includegraphics[width=\textwidth]{images/tree-ginkgo.png}
			\caption{A preprocessed Ginkgo tree}
			\label{fig:figure2}
		\end{minipage}
		\begin{minipage}{0.33\textwidth}
			\centering
			\includegraphics[width=\textwidth]{images/tree-quercus.png}
			\caption{A preprocessed Quercus tree}
			\label{fig:figure2}
		\end{minipage}
	\end{figure}
\end{frame}

\begin{frame}{Sofware and libraries: CGAL}
	\Large
	\begin{figure}[H]
		\centering
		\includegraphics[width=0.8\textwidth]{images/logo-cgal.png}
	\end{figure}
	\begin{center}
	  \Large Open source software library for \textbf{computational geometry algorithms}
	\end{center}
  \end{frame}

  \begin{frame}{Tree modeling: Alpha Wrapping}
	\begin{figure}[H]
	  \centering
		  \centering
		  \includegraphics[width=\textwidth]{images/alpha-wrapping-bike.jpg}
		  \captionsetup{font={scriptsize}}
		  \caption{Different LOD of the Alpha Wrapping of a bike\cite{cgal_alpha_wrapper}}
  \end{figure}
  \end{frame}


  \begin{frame}{Tree modeling: Alpha Wrapping}
	\Large
	\textcolor{red}{\textbf{Input:}}
	  \begin{itemize}
	  \item  3D model with possible defects
	  \end{itemize}
	  \textcolor{red}{\textbf{Output:} }
	  \begin{itemize}
		\item Water-tight mesh
		\item No self-intersections
		\item Strictly enclosing the input
		\item Well shaped triangles
	  \end{itemize}
  \end{frame}


  \begin{frame}{Tree modeling: Alpha Wrapping}
	\Large
	\begin{figure}[H]
	  \centering
	  \includegraphics[width=\textwidth]{images/alpha-wrapping-ball.jpg}
	  \captionsetup{font={scriptsize}}
	  \caption{Alpha Wrapping in 2D with Offset and different Alpha parameters}
  \end{figure}
  \end{frame}


  \begin{frame}{Tree modeling: Alpha Wrapping}
	\Large
	\href{https://youtu.be/xIIDolWCrgU}{video link}
	\begin{center}
	  \movie[width=1\textwidth,height=0.8\textheight,poster,showcontrols]{}
	  {images/alpha-wrapping.mp4}
	\end{center}
  \end{frame}

\begin{frame}{Tree Modeling: lod 1,2 and 3}
	\begin{figure}[h]
		\centering
		\begin{minipage}{0.3\textwidth}
			\centering
			\includegraphics[width=\textwidth]{images/tree-cone_lod1.png}
			\caption{A wrapped conifer tree for LOD 1}
			\label{fig:figure1}
		\end{minipage}\hfill
		\begin{minipage}{0.3\textwidth}
			\centering
			\includegraphics[width=\textwidth]{images/tree-cone_lod2.png}
			\caption{A wrapped conifer tree for LOD 2}
			\label{fig:figure2}
		\end{minipage}
		\begin{minipage}{0.33\textwidth}
			\centering
			\includegraphics[width=\textwidth]{images/tree-cone_lod3.png}
			\caption{A wrapped conifer tree for LOD 3}
			\label{fig:figure2}
		\end{minipage}
	\end{figure}
\end{frame}

\begin{frame}{Tree Modeling: lod 1,2 and 3}
	\begin{figure}[h]
		\centering
		\begin{minipage}{0.3\textwidth}
			\centering
			\includegraphics[width=\textwidth]{images/tree-oval_lod1.png}
			\caption{A wrapped Ginkgo tree for LOD 1}
			\label{fig:figure1}
		\end{minipage}\hfill
		\begin{minipage}{0.3\textwidth}
			\centering
			\includegraphics[width=\textwidth]{images/tree-oval_lod2.png}
			\caption{A wrapped Ginkgo tree for LOD 2}
			\label{fig:figure2}
		\end{minipage}
		\begin{minipage}{0.33\textwidth}
			\centering
			\includegraphics[width=\textwidth]{images/tree-oval_lod3.png}
			\caption{A wrapped Ginkgo tree for LOD 3}
			\label{fig:figure2}
		\end{minipage}
	\end{figure}
\end{frame}

\begin{frame}{Tree Modeling: lod 1,2 and 3}
	\begin{figure}[h]
		\centering
		\begin{minipage}{0.3\textwidth}
			\centering
			\includegraphics[width=\textwidth]{images/tree-round_lod1.png}
			\caption{A wrapped Quercus tree for LOD 1}
			\label{fig:figure1}
		\end{minipage}\hfill
		\begin{minipage}{0.3\textwidth}
			\centering
			\includegraphics[width=\textwidth]{images/tree-round_lod2.png}
			\caption{A wrapped Quercus tree for LOD 2}
			\label{fig:figure2}
		\end{minipage}
		\begin{minipage}{0.33\textwidth}
			\centering
			\includegraphics[width=\textwidth]{images/tree-round_lod3.png}
			\caption{A wrapped Quercus tree for LOD 3}
			\label{fig:figure2}
		\end{minipage}
	\end{figure}
\end{frame}


\begin{frame}{Tree Modelings}
	\centering
    \textbf{Used Alpha Values for Each LOD}
    \begin{tabular}{|c|c|c|c|c|}
        \hline
        Tree & LOD 0 & LOD 1 & LOD 2 & LOD 3 \\
        \hline
        Alpha & Nan & 20 & 50 & 100 \\
        \hline
    \end{tabular}

    \vspace{1em} % Adjust vertical space between tables as needed

    \textbf{Number of Faces for Each LOD}
    \begin{tabular}{|c|c|c|c|c|}
        \hline
        Tree & LOD 0 & LOD 1 & LOD 2 & LOD 3 \\
        \hline
        Trunk & 28 & 28 & 28 & 28 \\
        Cone & 72 & 894 & 6038 & 34260 \\
        Oval & 52 & 1260 & 9254 & 44942 \\
        Round & 30 & 1198 & 10152 & 45164 \\
        \hline
    \end{tabular}
\end{frame}

\begin{frame}{Tree Modeling: tagging leaves}
	\begin{figure}
		\centering
		\includegraphics[width=0.5\textwidth]{images/ginkgo_tagged.png}
		\caption{A tree with 4 different markers on the leaves}
		\label{fig:figure1}
	\end{figure}
\end{frame}

\begin{frame}{Tree Modeling: tree's scaling}
	\begin{figure}[h]
		\centering
		\begin{minipage}{0.49\textwidth}
			\centering
			\includegraphics[width=\textwidth]{images/foliage_arrows.png}
			\caption{Foliage mesh}
			\label{fig:figure1}
		\end{minipage}\hfill
		\begin{minipage}{0.49\textwidth}
			\centering
			\includegraphics[width=0.5\textwidth]{images/trunk_arrows.png}
			\caption{Trunk mesh}
			\label{fig:figure2}
		\end{minipage}
	\end{figure}
\end{frame}


\begin{frame}{Tree Modeling: Mercator's projection}
	\begin{figure}[H]
	  \centering
	  \includegraphics[width=\textwidth]{images/mercator.jpg}
	  \captionsetup{font={scriptsize}}
	  \caption{Mercator's projection\cite{img:mercator}}
  \end{figure}

  \end{frame}

  \begin{frame}{Tree Modeling: Mercator's projection}
	\Large
	A(latitude, longitude) = A($\phi$, $\lambda$),\\
	\begin{equation}
	  \text{projection} \Longrightarrow \quad
	  \left\{
	  \begin{array}{l}
		  x =  \lambda - \lambda_{0} \\
		  y =  \ln(\tan(\frac{\pi}{4} + \frac{\phi}{2}))
	  \end{array}
	  \right.
	\end{equation}
	\vfill
	,where $\lambda_{0}$ is the center of the map
  \end{frame}


  \begin{frame}{Tree Modeling: Mercator's projection}
	\Large
	\begin{figure}[H]
	  \centering
	  \begin{minipage}{0.49\textwidth}
		  \centering
		  \includegraphics[width=0.8\textwidth]{images/WGS84-earth-radius.png}
		  \captionsetup{font={scriptsize}}
		  \caption{Earth as an ellipsoid\cite{mercator-proj}}
	  \end{minipage}\hfill
	  \begin{minipage}{0.49\textwidth}
		  \centering
		  \includegraphics[width=0.8\textwidth]{images/WGS84-frame.png}
		  \captionsetup{font={scriptsize}}
		  \caption{WGS 84 reference frame\cite{mercator-proj}}
	  \end{minipage}
  \end{figure}

  \textit{WGS84toCartesian.hpp} $\Longrightarrow$ \textbf{GPS} to \textbf{Cartesian}
  \end{frame}

  \begin{frame}{Tree Modeling: foliage and trunk placement}
	\begin{figure}
		\centering
		\includegraphics[width=0.5\textwidth]{images/trunk_cut.png}
		\caption{.3D model of a tree with a trunk that is too long}
		\label{fig:figure1}
	\end{figure}
\end{frame}

\begin{frame}{Tree Modeling: tree's placement}
  \begin{figure}[H]
	\centering
	\begin{minipage}{0.49\textwidth}
		\centering
		\includegraphics[width=1\textwidth]{images/republic-comparaison.png}
		\captionsetup{font={scriptsize}}
		\caption{Republic square with LOD 1 trees}
	\end{minipage}\hfill
	\begin{minipage}{0.49\textwidth}
		\centering
		\includegraphics[width=1\textwidth]{images/ovt-republic.png}
		\captionsetup{font={scriptsize}}
		\caption{Republic square trees from Overpass turbo\cite{overpass-turbo}}
	\end{minipage}
  \end{figure}
\end{frame}

\begin{frame}{Tree Modeling: tree's placement}
	\begin{figure}[H]
		  \centering
		  \includegraphics[width=\textwidth]{images/republic-side-view.png}
		  \captionsetup{font={scriptsize}}
		  \caption{Republic square with LOD 1 trees}
	\end{figure}
\end{frame}

\begin{frame}{Tree Modeling: tree's altitude}
	\begin{figure}
		\centering
		\includegraphics[width=\textwidth]{images/raycasting.png}
		\caption{Raycasting of a tree in Grenoble, France}
		\label{fig:figure1}
	\end{figure}
\end{frame}

\begin{frame}{Raycasting: BVH}
	\begin{figure}
		\centering
		\includegraphics[width=\textwidth]{images/whitted_rubin_bvh.jpg}
		\caption{Image based on A 3-Dimensional Representation for Fast Rendering of Complex Scenes, by Rubin and Whitted. \cite{how-to-bvh}}
		\label{fig:figure1}
	\end{figure}
\end{frame}

\begin{frame}{Tree Modeling: mesh merging}
	\begin{figure}
		\centering
		\includegraphics[width=0.45\textwidth]{images/plants_wall.jpg}
		\caption{A plant going through a wall}
		\label{fig:figure1}
	\end{figure}
\end{frame}

\begin{frame}{Tree Modeling: mesh merging}
	\begin{figure}[h]
		\begin{minipage}{0.49\textwidth}
			\centering
			\includegraphics[width=\textwidth]{images/autorefine_before.png}
			\caption{Autorefine before}
			\label{fig:figure1}
		\end{minipage}\hfill
		\begin{minipage}{0.49\textwidth}
			\centering
			\includegraphics[width=\textwidth]{images/autorefine_after.png}
			\caption{Autorefine after}
			\label{fig:figure2}
		\end{minipage}
	\end{figure}
\end{frame}

\begin{frame}{Tree Modeling: thread parallelization}
	\begin{figure}
		\centering
		\includegraphics[width=0.8\textwidth]{images/thread_parallelization.PNG}
		\caption{Parallelization of the tree generation process}
		\label{fig:figure1}
	\end{figure}
\end{frame}

\begin{frame}{Tree Modeling: thread parallelization autorefine}
	\begin{figure}
		\centering
		\includegraphics[width=0.5\textwidth]{images/stras_cutted.png}
		\caption{Parallelization of the mesh refinement process}
		\label{fig:figure1}
	\end{figure}
\end{frame}

\begin{frame}{Results: Model Integration}
	\begin{figure}
		\centering
		\includegraphics[width=\textwidth]{images/integration_buildings_only.png}
		\caption{3D Model of Strasbourg city center, buildings only}
		\label{fig:figure1}
	\end{figure}
\end{frame}

\begin{frame}{Results: Model Integration}
	\begin{figure}
		\centering
		\includegraphics[width=\textwidth]{images/integration_trees_only.png}
		\caption{3D Model of Strasbourg city center, trees only}
		\label{fig:figure1}
	\end{figure}
\end{frame}

\begin{frame}{Results: Model Integration}
	\begin{figure}
		\centering
		\includegraphics[width=\textwidth]{images/integration_buildings_n_trees.png}
		\caption{3D Model of Strasbourg city center, buildings and trees}
		\label{fig:figure1}
	\end{figure}
\end{frame}

\begin{frame}{Results: Model Integration}
	\begin{figure}
		\centering
		\includegraphics[width=\textwidth]{images/integration_republic.png}
		\caption{3D Model of Strasbourg city center, buildings and vegetation, with a focus on Republic Square, LOD 0}
		\label{fig:figure1}
	\end{figure}
\end{frame}

\begin{frame}{Results: Model Integration}
	\begin{figure}
		\centering
		\includegraphics[width=\textwidth]{images/integration_republic_lod3.png}
		\caption{3D Model of Strasbourg city center, buildings and vegetation, with a focus on Republic Square, LOD 3}
		\label{fig:figure1}
	\end{figure}
\end{frame}


\begin{frame}{Results: Performance}
	\Large
	\begin{figure}[H]
	  \centering
	  \includegraphics[width=0.7\textwidth]{images/ovt-bbox1.png}
	  \captionsetup{font={scriptsize}}
	  \caption{Bounding Box 1: 153.7 m², 12 trees}
  \end{figure}
  \end{frame}

\begin{frame}{Results: Performance}
\Large
\begin{figure}[H]
	\centering
	\includegraphics[width=0.7\textwidth]{images/ovt-bbox2.png}
	\captionsetup{font={scriptsize}}
	\caption{Bounding Box 2: 384.0 m², 71 trees}
\end{figure}
\end{frame}

\begin{frame}{Results: Performance}
\Large
\begin{figure}[H]
	\centering
	\includegraphics[width=0.7\textwidth]{images/ovt-bbox3.png}
	\captionsetup{font={scriptsize}}
	\caption{Bounding Box 3: 626.1 m², 254 trees}
\end{figure}
\end{frame}

\begin{frame}{Results: Performance}
\Large
\begin{figure}[H]
	\centering
	\includegraphics[width=0.7\textwidth]{images/ovt-bbox4.png}
	\captionsetup{font={scriptsize}}
	\caption{Bounding Box 4: 808.4 m², 513 trees}
\end{figure}
\end{frame}

\begin{frame}{Results: Performance}
\Large
Without LOD 3
\begin{figure}[H]
	\centering
	\includegraphics[width=0.9\textwidth]{images/perf_lod_faces.png}
	\captionsetup{font={scriptsize}}
	\caption{Relationship between the number of faces and the number of trees}
\end{figure}
\end{frame}

\begin{frame}{Results: Performance}
	\Large
	With LOD 3
	\begin{figure}[H]
		\centering
		\includegraphics[width=\textwidth]{images/perf_lod_faces_lod3.png}
		\captionsetup{font={scriptsize}}
		\caption{Relationship between the number of faces and the number of trees}
	\end{figure}
\end{frame}

\begin{frame}{Results: Performance}
	\Large
	\begin{figure}[H]
		\centering
		\includegraphics[width=1\textwidth]{images/perf_autorefine_ON_OFF.png}
		\captionsetup{font={scriptsize}}
		\caption{Comparison of the execution time with and without the autorefine method}
	\end{figure}
\end{frame}

\begin{frame}{Results: Performance}
	\Large
	\begin{figure}[H]
		\centering
		\includegraphics[width=1\textwidth]{images/perf_async_improv.png}
		\captionsetup{font={scriptsize}}
		\caption{Thread parallelization improvement}
	\end{figure}
\end{frame}

\begin{frame}{Prospects}
	\begin{itemize}
		\item Fixing the tree's altitude
	\end{itemize}

	\begin{figure}
		\centering
		\includegraphics[width=\textwidth]{images/grenoble_fail.png}
		\caption{Grenoble elevation fail}
		\label{fig:figure1}
	\end{figure}
\end{frame}

\begin{frame}{Prospects}
	\begin{itemize}
		\item Parallelize the computation of the `autorefine` method
	\end{itemize}

	\begin{figure}
		\centering
		\includegraphics[width=0.5\textwidth]{images/stras_cutted.png}
		\caption{.Terrain mesh splitted into smaller parts, the red lines represent the boundaries of the parts and should not be parallelized}
		\label{fig:figure1}
	\end{figure}
\end{frame}

\begin{frame}{Prospects}
	\begin{itemize}
		\item Improve the handling of missing tree data
	\end{itemize}

	\begin{figure}
		\centering
		\includegraphics[width=0.7\textwidth]{images/ovt-node-missing-data.png}
		\caption{An Overpass Turbo Node with missing metadata}
		\label{fig:figure1}
	\end{figure}
\end{frame}

\begin{frame}{Prospects}
	\begin{itemize}
		\item Shading calculation
	\end{itemize}

	\begin{figure}
		\centering
		\includegraphics[width=0.6\textwidth]{images/cathedrale_shading.png}
		\caption{The shadow cast by the Cathedral of Strasbourg }
		\label{fig:figure1}
	\end{figure}
\end{frame}


\begin{frame}{Conclusion}
    \begin{itemize}
        \item \textbf{Project Goal:} Enhanced urban modeling by integrating 3D tree models into urban environments.
        \item \textbf{Key Achievements:}
        \begin{itemize}
            \item Extracted tree data from OpenStreetMap.
            \item Generated 3D tree models using CGAL and Gmsh.
            \item Integrated models into existing terrain meshes.
            \item Utilized different Levels of Detail (LOD) for flexible modeling.
        \end{itemize}
        \item \textbf{Technical Highlights:}
        \begin{itemize}
            \item Robust data acquisition and model generation techniques.
            \item Successful scalability and performance benchmarks.
            \item Addressed missing data with default tree height values.
        \end{itemize}
        \item \textbf{Future Directions:}
        \begin{itemize}
            \item Incorporate tree elevations, parallelization, and advanced shading for improved modeling.
        \end{itemize}
    \end{itemize}
\end{frame}


\begin{frame}{The end}
    \Large
    \centering
    \textbf{Thank you for your attention!} \\
    \vspace{1em}
    \includegraphics[height=1cm]{images/party-emoji.png} \\
    \vspace{1em}
    \textbf{Any questions?} \\
    \vspace{2em}
    \small
    \textit{Contact Information:} \\
    pierre.antoine.senger@gmail.com \\
    github.com/PA-Senger
\end{frame}

\nocite{*}
\bibliographystyle{unsrt}
\bibliography{references}

\end{document}